\Definition{}{
	An \Keyword{argument} can be defined as a \Domain{set} of \Domain{statement}s, one of which is the \Domain{conclusion} (the thing \Domain{argue}-ed for) and the others are the \Domain{premise}s (\Domain{reason}s for \Domain{accept}ing the \Domain{conclusion}).
	The \Domain{sentence}s in an \Domain{argument} must \Domain{express} \Domain{statement}s—that is, say something that is either \Domain{true} or \Domain{false}.
	But even if every \Domain{sentence} in a \Domain{group} of \Domain{sentence}s \Domain{express}es a \Domain{statement}, the result is not necessarily an \Domain{argument}. The \Domain{statement}s must be \Domain{relate}d to one another \Domain{statement}s in the appropriate way. Something must be \Domain{argue}-ed for (the \Domain{conclusion}), and there must be \Domain{reason}s (the \Domain{premise}s) for \Domain{accept}ing the \Domain{conclusion}.
}

\Property{}{
	An \Domain{argument} may be embedded in other \Domain{argument}s.
}{${\proving}$}

\Property{}{
	\Domain{Argument} is a kind of \Domain{reasoning}.
}{${\proving}$}

\Property{}{
	\Domain{Argument} consists of \Domain{statement}s.
}{${\proving}$}

\Property{}{
	Not just any group of \Domain{sentence}s makes an \Domain{argument}.
}{${\proving}$}

\Property{}{
	A \Domain{statement} is best thought of as something that can be \Domain{express}ed using a \Domain{declarative} \Domain{sentence}.
}{${\proving}$}

\Definition{}{
	\Domain{Statement}s in an \Domain{argument} that give \Domain{reason}s for \Domain{accept}ing a \Domain{statement} are said to be \Keyword{premises} of an \Domain{argument}.
}

\Definition{}{
	The \Domain{argument} is called the \Domain{argument}'s \Keyword{conclusion}.
}

\Property{}{
	An \Domain{argument} may have any number of \Domain{premise}s (even only one!)
}{${\proving}$}

\Definition{}{
	To determine whether or not an \Domain{argument} is \Keyword{valid}, one must ask whether there are any possible circumstances under which the \Domain{premise}s could all be \Domain{true} and yet the \Domain{conclusion} be \Domain{false}. If not, the \Domain{argument} is \Domain{valid}.
}

\Property{}{
	The \Domain{truth} of the \Domain{premise}s of a \Domain{valid} \Domain{argument} guarantees the \Domain{truth} of its \Domain{conclusion}.
}{${\proving}$}

\Property{}{
	It is impossible for the \Domain{conclusion} of a \Domain{valid} \Domain{argument} to be \Domain{false} if all its \Domain{premise}s are \Domain{true}.
}{${\proving}$}

\Property{}{
	In a \Domain{valid} \Domain{argument}, if \Domain{premise}s are both \Domain{true}, then the \Domain{conclusion} must be \Domain{true} also. But are both \Domain{premise}s true? We have no way of knowing.
}{${\proving}$}

\Definition{}{
	If it is possible for the \Domain{premise}s to be \Domain{true} and the \Domain{conclusion} \Domain{false}, the \Domain{argument} is \Keyword{invalid}.
}

\Property{}{
	An \Domain{invalid} \Domain{argument} is simply an \Domain{argument} that is not \Domain{valid}.
}{${\proving}$}

\Property{}{
    \Domain{Deduction} and \Domain{induction} are commonly thought to be the cornerstones of good \Domain{reasoning}.
}{${\proving}$}

\Definition{}{
	The fundamental logical property of a \Keyword{deductive}ly \Domain{valid} \Domain{argument} is this: If all its \Domain{premise}s are \Domain{true}, then its \Domain{conclusion} must be \Domain{true}. In other words, an \Domain{argument} is \Domain{valid} if it is impossible for all its \Domain{premise}s to be \Domain{true} and yet its \Domain{conclusion} be \Domain{false}.
}

\Property{}{
	The \Domain{truth} of the \Domain{premise}s of a \Domain{deductive}ly \Domain{valid} \Domain{argument} guarantees the \Domain{truth} of its \Domain{conclusion}.
}{${\proving}$}

\Property{}{
	It is revealing to notice that in a typical case the information contained in the \Domain{conclusion} of a \Domain{deductive}ly \Domain{valid} \Domain{argument} is already "contained” in its \Domain{premise}s. It is usually contained in the \Domain{premise}s implicitly (along with other information not contained in the \Domain{conclusion})
}{${\proving}$}

\Definition{}{
	In addition to \Domain{deductive} \Domain{argument}s, there are also \Keyword{inductive} \Domain{argument}s.
}

\Property{}{
	\Domain{Inductive} \Domain{argument}s differ from \Domain{deductive}ly \Domain{valid} \Domain{argument}s in having \Domain{conclusion}s that go beyond what is contained in their \Domain{premise}s.
}{${\proving}$}

\Property{}{
	The crucial difference between \Domain{inductive} \Domain{strength} and \Domain{deductive} \Domain{validity} is that it is possible for the \Domain{premise}s of a \Domain{strong} \Domain{inductive} \Domain{argument} to be \Domain{true} and yet the \Domain{conclusion} be \Domain{false}.
}{${\proving}$}

\Property{}{
	The basic idea behind \Domain{inductive} \Domain{reasoning} is that of learning from \Domain{experience}. We notice \Domain{pattern}s, \Domain{resemblance}s, or other kinds of \Domain{regularity}-s in our \Domain{experience}s
}{${\proving}$}

\Property{}{
	The great virtue of \Domain{inductive} \Domain{reasoning} is that it provides us with a way of \Domain{reasoning} to genuinely new beliefs, and not just to psychologically new ones that were implicit in what we already knew, as in the case of \Domain{valid} \Domain{deduction}s. However, this benefit is purchased at the cost of an increase in the possibility of \Domain{error}. 
}{${\proving}$}

\Definition{}{
	Good \Domain{inductive} \Domain{argument}s are said to be \Domain{inductive}ly \Keyword{strong}.
}

\Property{}{
	Whereas \Domain{true} \Domain{premise}s in a \Domain{valid} \Domain{argument} \Domain{guarantee} the \Domain{truth} of the \Domain{conclusion}, \Domain{true} \Domain{premise}s in a \Domain{strong} \Domain{inductive} \Domain{argument} make the \Domain{conclusion} \Domain{likely} or \Domain{probable}.
}{${\proving}$}

\Property{}{
	Although an \Domain{inductive}ly \Domain{strong} \Domain{argument} does not guarantee that if its \Domain{premise}s are \Domain{true}, then its \Domain{conclusion} also will be \Domain{true}, it does make its \Domain{conclusion} more \Domain{probable}.
}{${\proving}$}

\Definition{}{
	The expression "\Keyword{probable}-ility \Keyword{argument}” is sometimes applied to \Domain{inductive} \Domain{argument}s.
}

\Definition{}{
	Argument \Keyword{form} is, informally, the logical structure of an argument. 
}



\Property{}{
	A \Domain{strong} \Domain{induction} may contain all \Domain{true} \Domain{premise}s and yet have a \Domain{false} \Domain{conclusion}.
}{${\proving}$}

\Property{}{
	Unlike \Domain{valid}-ity, \Domain{inductive} \Domain{strong}-ength comes in \Domain{degree}s. It makes no sense to speak of one \Domain{deductive} \Domain{argument} as being "more \Domain{valid}" than another. All \Domain{deductive} \Domain{argument}s are either \Domain{valid} or \Domain{invalid}. But it does make sense to describe one \Domain{argument} as being \Domain{inductive}ly \Domain{stronge}r than another.
}{${\proving}$}

\Property{}{
	Nevertheless, it is clear that if the \Domain{premise}s are \Domain{true}, then the \Domain{conclusion} must be \Domain{true}. We know this from the \Domain{form} of the \Domain{argument} and not because of its \Domain{content}; its \Domain{form} makes this \Domain{argument} a \Domain{valid} \Domain{argument} \Domain{form}.
}{${\proving}$}

\Property{}{
	Any argument having the same (\Domain{valid}) \Domain{form} is \Domain{deductive}ly \Domain{valid}. Of course, there are many other \Domain{valid} \Domain{argument} \Domain{form}s.
}{${\proving}$}

\Property{}{
	\Domain{Logic} is concerned primarily with \Domain{argument} \Domain{form}s, and only secondarily with \Domain{argument}s, for all \Domain{argument}s that have a \Domain{valid} \Domain{argument} \Domain{form} are \Domain{valid}. The \Domain{form}, not the \Domain{content}, of the preceding \Domain{argument}s makes it impossible for them to have all \Domain{true} \Domain{premise}s and a \Domain{false} \Domain{conclusion}. 
}{${\proving}$}

\Property{}{
	In general, \Domain{argument}s have several \Domain{form}s; any \Domain{argument} with at least one \Domain{valid} \Domain{form} is \Domain{valid}.
}{${\proving}$}

