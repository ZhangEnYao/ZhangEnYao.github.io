\Property{}{
	\Domain{Argument} consists of \Domain{statement}s.
	An \Domain{argument} may have any number of \Domain{premise}s (even only one!) and may be embedded in other \Domain{argument}s.
}{}

\Property{}{
	Not just any group of \Domain{sentence}s makes an \Domain{argument}.
}{}

\Property{}{
    \Domain{Deduction} and \Domain{induction} are commonly thought to be the cornerstones of good \Domain{reasoning}.
}{}

\Property{}{
	The \Domain{truth} of the \Domain{premise}s of a \Domain{valid} \Domain{argument} guarantees the \Domain{truth} of its \Domain{conclusion}.
}{}

\Property{}{
	The fundamental logical property of a \Domain{deductive}ly \Domain{valid} \Domain{argument} is this: If all its \Domain{premise}s are \Domain{true}, then its \Domain{conclusion} must be \Domain{true}. In other words, an \Domain{argument} is \Domain{valid} if it is impossible for all its \Domain{premise}s to be \Domain{true} and yet its \Domain{conclusion} be \Domain{false}.
	It is impossible for the \Domain{conclusion} of a \Domain{valid} \Domain{argument} to be \Domain{false} if all its \Domain{premise}s are \Domain{true}.
	The \Domain{truth} of the \Domain{premise}s of a \Domain{deductive}ly \Domain{valid} \Domain{argument} guarantees the \Domain{truth} of its \Domain{conclusion}.
}{}

\Property{}{
	It is revealing to notice that in a typical case the \Domain{information} contained in the \Domain{conclusion} of a \Domain{deductive}ly \Domain{valid} \Domain{argument} is already "contained” in its \Domain{premise}s.
	It is usually contained in the \Domain{premise}s implicitly (along with other \Domain{information} not contained in the \Domain{conclusion})
}{}

\Property{}{
	\Domain{Inductive} \Domain{argument}s differ from \Domain{deductive}ly \Domain{valid} \Domain{argument}s in having \Domain{conclusion}s that go beyond what is contained in their \Domain{premise}s.
}{}

\Property{}{
	The crucial difference between \Domain{inductive} \Domain{strength} and \Domain{deductive} \Domain{validity} is that it is possible for the \Domain{premise}s of a \Domain{strong} \Domain{inductive} \Domain{argument} to be \Domain{true} and yet the \Domain{conclusion} be \Domain{false}.
	Whereas \Domain{true} \Domain{premise}s in a \Domain{valid} \Domain{argument} \Domain{guarantee} the \Domain{truth} of the \Domain{conclusion}, \Domain{true} \Domain{premise}s in a \Domain{strong} \Domain{inductive} \Domain{argument} make the \Domain{conclusion} \Domain{likely} or \Domain{probable}.
	Although an \Domain{inductive}ly \Domain{strong} \Domain{argument} does not guarantee that if its \Domain{premise}s are \Domain{true}, then its \Domain{conclusion} also will be \Domain{true}, it does make its \Domain{conclusion} more \Domain{probable}.
}{}

\Property{}{
	The basic idea behind \Domain{inductive} \Domain{reasoning} is that of learning from \Domain{experience}. We notice \Domain{pattern}s, \Domain{resemblance}s, or other kinds of \Domain{regularity}-s in our \Domain{experience}s
}{}

\Property{}{
	The great virtue of \Domain{inductive} \Domain{reasoning} is that it provides us with a way of \Domain{reasoning} to genuinely new beliefs, and not just to psychologically new ones that were implicit in what we already knew, as in the case of \Domain{valid} \Domain{deduction}s. However, this benefit is purchased at the cost of an increase in the possibility of \Domain{error}. 
	}{}

\Property{}{
	A \Domain{strong} \Domain{induction} may contain all \Domain{true} \Domain{premise}s and yet have a \Domain{false} \Domain{conclusion}.
}{}

\Property{}{
	Unlike \Domain{validity}, \Domain{inductive} \Domain{strength} comes in \Domain{degree}s. It makes no sense to speak of one \Domain{deductive} \Domain{argument} as being "more \Domain{valid}" than another. All \Domain{deductive} \Domain{argument}s are either \Domain{valid} or \Domain{invalid}. But it does make sense to describe one \Domain{argument} as being \Domain{inductive}ly \Domain{stronge}r than another.
}{}

\Property{}{
	In a \Domain{valid} \Domain{argument}, if \Domain{premise}s are both \Domain{true}, then the \Domain{conclusion} must be \Domain{true} also. But are both \Domain{premise}s true? We have no way of knowing.
}{}

\Property{}{
	Nevertheless, it is clear that if the \Domain{premise}s are \Domain{true}, then the \Domain{conclusion} must be \Domain{true}. We know this from the \Domain{form} of the \Domain{argument} and not because of its \Domain{content}; its \Domain{form} makes this \Domain{argument} a \Domain{valid} \Domain{argument} \Domain{form}.
	Any argument having the same (\Domain{valid}) \Domain{form} is \Domain{deductive}ly \Domain{valid}. Of course, there are many other \Domain{valid} \Domain{argument} \Domain{form}s.
	\Domain{Logic} is concerned primarily with \Domain{argument} \Domain{form}s, and only secondarily with \Domain{argument}s, for all \Domain{argument}s that have a \Domain{valid} \Domain{argument} \Domain{form} are \Domain{valid}. The \Domain{form}, not the \Domain{content}, of the preceding \Domain{argument}s makes it impossible for them to have all \Domain{true} \Domain{premise}s and a \Domain{false} \Domain{conclusion}. 
}{}

\Property{}{
	In general, \Domain{argument}s have several \Domain{form}s; any \Domain{argument} with at least one \Domain{valid} \Domain{form} is \Domain{valid}.
}{}
