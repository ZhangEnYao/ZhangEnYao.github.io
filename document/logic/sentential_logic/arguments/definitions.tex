\Definition{}{
	\Domain{Argument} is a kind of \Domain{reasoning}.
	An \Domain{argument} can be defined as a set of \Domain{statement}s, one of which is the \Domain{conclusion} (the thing \Domain{argue}-ed for) and the others are the \Domain{premise}s (\Domain{reason}s for accepting the \Domain{conclusion}).
}

\Definition{}{
	\Domain{Statement}s in an \Domain{argument} that give \Domain{reason}s for accepting the \Domain{argument}'s \Domain{conclusion} are said to be \Domain{premises} of an \Domain{argument}.
}

\Definition{}{
	The \Domain{sentence}s in an \Domain{argument} must express \Domain{statement}s—that is, say something that is either \Domain{true} or \Domain{false}.
	But even if every \Domain{sentence} in a group of \Domain{sentence}s expresses a \Domain{statement}, the result is not necessarily an \Domain{argument}. The \Domain{statement}s must be related to one another \Domain{statement}s in the appropriate way. Something must be \Domain{argue}-ed for (the \Domain{conclusion}), and there must be \Domain{reason}s (the \Domain{premise}s) for accepting the \Domain{conclusion}.
}

\Definition{}{
	A \Domain{statement} is best thought of as something that can be expressed using a \Domain{declarative} \Domain{sentence}.
}

\Definition{}{
	To determine whether or not an \Domain{argument} is \Domain{valid}, one must ask whether there are any possible circumstances under which the \Domain{premise}s could all be \Domain{true} and yet the \Domain{conclusion} be \Domain{false}. If not, the \Domain{argument} is \Domain{valid}.
}

\Definition{}{
	If it is possible for the \Domain{premise}s to be \Domain{true} and the \Domain{conclusion} \Domain{false}, the \Domain{argument} is \Domain{invalid}.
	An \Domain{invalid} \Domain{argument} is simply an \Domain{argument} that is not \Domain{valid}.
}

\Definition{}{
	In addition to \Domain{deductive} \Domain{argument}s, there are also \Domain{inductive} \Domain{argument}s.
}

\Definition{}{
	Good \Domain{inductive} \Domain{argument}s are said to be \Domain{inductive}ly \Domain{strong}.
}

\Definition{}{
	The expression "\Domain{probability} \Domain{argument}” is sometimes applied to \Domain{inductive} \Domain{argument}s.
}
